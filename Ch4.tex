%!TEX root =  main.tex
\chapter*{Chapter 4 Stochastic Bandits}
\label{sec:second}

\noindent\textbf{4.1}
By definition
\begin{align}
R_n(\pi,v) &=n\mu^*(v)-\mathbb{E}[\sum^n_{t=1}X_t]\notag\\
&=\sum^n_{t=1}\mu^*(v)-\sum^n_{t=1}\mathbb{E}[X_t]\notag\\
&=\sum^n_{t=1}[\mu^*-\mu_{A_t}]\notag
\end{align}
\begin{enumerate}[(a)]
    \item $\mu^*=\max\mu_a\ge\mu_{A_t} \Rightarrow R_n(\pi,v)=\sum^n_{t=1}[\mu^*-\mu_{A_t}]\ge0$.

    \item If $\pi$ choose $A_t \in \arg\max_a\mu_a$ for all $t\in[n]$ $\Rightarrow \sum^n_{t=1}[\mu^*-\mu_{A_t}]=0$.

\item If $R_n(\pi,v)=0$ for some policy $\pi$ ,then $A_t \in \arg\max_a\mu_a \Rightarrow \mathbb{P}(\mu_{A_t}=\mu^*)=1$.
\end{enumerate}



\noindent\textbf{4.2}(Uniqueness of law) Prove Proposition 4.8. 
\begin{proposition}
Suppose that $\bP$ and $\bQ$ are probability measures on an arbitrary measurable space $(\Omega,\cF)$ and $A_1,X_1,...,A_n,X_n$ are random variables on $\Omega$, where $A_t\in [k]$ and $X_t\in\RR$. If both $\bP$ and $\bQ$ satisfy (a) and (b), then the law of the outcome under $\bP$ is the same as under $\bQ$: 
$$\bP_{A_1,X_1,...,A_n,X_n}=\bQ_{A_1,X_1,...,A_n,X_n}\,.$$
Recall the condition:
\begin{enumerate}
    \item[(a)] the conditional distribution of action $A_t$ given $A_1,X_1,...,A_{t-1},X_{t-1}$ is $\pi_t(\cdot \mid A_1,X_1,...,A_{t-1},X_{t-1})$ almost surely.
    \item[(b)] the conditional distribution of reward $X_t$ given $A_1,X_1,...,A_{t}$ is $P_{A_t}$ almost surely.
\end{enumerate}
\end{proposition}

\begin{proof}
    \begin{align*}
        \bP_{A_1,X_1,...,A_n,X_n} &= \bP(A_1) \bP(X_1\mid A_1)\ldots \bP(A_n \mid A_1,X_1,...,A_{n-1},X_{n-1}) \bP(X_n \mid A_1,X_1,...,A_{n-1},X_{n-1},A_n) \\
        &= \pi_1(A_1) P_{A_1}\ldots \pi_n(A_n\mid A_1,X_1,...,A_{n-1},X_{n-1} ) P_{A_n}\\
        &= \bQ(A_1) \bQ(X_1\mid A_1)\ldots \bQ(A_n \mid A_1,X_1,...,A_{n-1},X_{n-1}) \bQ(X_n \mid A_1,X_1,...,A_{n-1},X_{n-1},A_n)\\
        &= \bQ_{A_1,X_1,...,A_n,X_n} \,.
    \end{align*}
\end{proof}
    





\noindent\textbf{4.3}
Denote $h_t = a_1, x_1, \ldots, a_t, x_t$.
\begin{enumerate}[(a)]
    \item According to the definition of conditional probability and marginal distribution, we have
    \begin{equation*}
    \begin{aligned}
        p_{v\pi}(a_n \mid h_{n - 1}) &= \frac{p_{v\pi}(h_{n-1}, a_n)}{p_{v\pi}(h_{n-1})}\\
        &= \frac{\int_\mathbb{R}p_{v\pi}(h_n) d x_n}{p_{v\pi}(h_{n-1})}\\
        &= \frac{\int_\mathbb{R}\prod_{t=1}^{n} \pi\left(a_{t} \mid h_{t-1}\right) p_{a_{t}}\left(x_{t}\right) d x_n}{p_{v\pi}(h_{n-1})}\\
        &= \frac{\prod_{t=1}^{n-1} \pi\left(a_{t} \mid h_{t-1}\right) p_{a_{t}}\left(x_{t}\right)}{p_{v\pi}(h_{n-1})} \int_\mathbb{R} \pi\left(a_{n} \mid h_{n-1}\right) p_{a_{n}}\left(x_{n}\right) d x_n\\
        &= \pi\left(a_{n} \mid h_{n-1}\right) \int_\mathbb{R} p_{a_{n}}\left(x_{n}\right) d x_n\\
        &= \pi\left(a_{n} \mid h_{n-1}\right)
    \end{aligned}
    \end{equation*}

    \item According to the definition of conditional probability and marginal distribution, we have
    \begin{equation*}
    \begin{aligned}
        p_{v\pi}(x_n \mid h_{n - 1}, a_n) &= \frac{p_{v\pi}(h_n)}{p_{v\pi}(h_{n-1}, a_n)}\\
        &= \frac{p_{v\pi}(h_n)}{\int_\mathbb{R} p_{v\pi}(h_{n}) d x_n}\\
        &= \frac{p_{v\pi}(h_n)}{\int_\mathbb{R} \left[\prod_{t=1}^{n} \pi\left(a_{t} \mid h_{t-1}\right) p_{a_{t}}\left(x_{t}\right)\right] d x_n}\\
        &= \frac{p_{v\pi}(h_n)}{\prod_{t=1}^{n-1} \pi\left(a_{t} \mid h_{t-1}\right) p_{a_{t}}\left(x_{t}\right)} \frac{1}{\int_\mathbb{R} \pi\left(a_{n} \mid h_{n-1}\right) p_{a_{n}}\left(x_{n}\right) d x_n}\\
        &= \pi\left(a_{n} \mid h_{n-1}\right) p_{a_{n}}\left(x_{n}\right) \frac{1}{\pi\left(a_{n} \mid h_{n-1}\right)}\\
        &= p_{a_{n}}\left(x_{n}\right)
    \end{aligned}
    \end{equation*}
\end{enumerate}

\noindent\textbf{4.4}
Denote $h_t = a_1, x_1, \ldots, a_t, x_t$.
The policy that mixes the policies can be defined as
$$\pi_{t}^{\circ}\left(a_{t} \mid h_{t-1}\right)=\frac{\sum_{\pi \in \Pi} p(\pi) \prod_{s=1}^{t} \pi_{s}\left(a_{s} \mid h_{s-1}\right)}{\sum_{\pi \in \Pi} p(\pi) \prod_{s=1}^{t-1} \pi_{s}\left(a_{s} \mid h_{s-1}\right)}$$.

By the definition of the canonical probability space and the product of probability kernels,
\begin{equation*}
    \begin{aligned}
    \mathbb{P}_{v \pi^{\circ}}(B) &= \sum_{a_{1}=1}^{k} \int_{\mathbb{R}} \cdots \sum_{a_{n}=1}^{k} \int_{\mathbb{R}} \mathbb{I}_{B}\left(h_{n}\right) v_{a_{n}}\left(d x_{n}\right) \pi_{n}^{\circ}\left(a_{n} \mid h_{n-1}\right) \cdots v_{a_{1}}\left(d x_{1}\right) \pi_{1}^{\circ}\left(a_{1}\right)\\
    &= \sum_{\pi \in \Pi} p(\pi) \sum_{a_{1}=1}^{k} \int_{\mathbb{R}} \cdots \sum_{a_{n}=1}^{k} \int_{\mathbb{R}} \mathbb{I}_{B}\left(h_{n}\right) v_{a_{n}}\left(d x_{n}\right) \pi_{n}\left(a_{n} \mid h_{n-1}\right) \cdots v_{a_{1}}\left(d x_{1}\right) \pi_{1}\left(a_{1}\right)\\
    &= \sum_{\pi \in \Pi} p(\pi) \mathbb{P}_{v \pi}(B),
    \end{aligned}
\end{equation*}
where the second equality follows by substituting the definition of $\pi_{n}^{\circ}$ and induction.
